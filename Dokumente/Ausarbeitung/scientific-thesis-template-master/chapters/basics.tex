% Grundlagen für Arbeit (Related Work), Sehapparat, Raytracing, Volumenrendering, Eyetracking, GPU Architektur (Warps usw.) %

\chapter{Grundlagen}
\label{chap:k2}
Dieses Kapitel beschäftigt sich mit den Grundlagen für diese Arbeit.
Der erste Abschnitt des Kapitels beschäftigt sich mit verwandten Arbeiten zum Thema Wahrnehmungsorientiertes Volumen-Rendering.
Der zweite Abschnitt handelt über die Grundlagen des menschlichen Sehapparates.
Hier werden die Fähigkeiten und Limitierungen der visuellen Wahrnehmung des Menschen diskutiert.
In Abschnitt drei, wird die Funktionsweise von Raytracing erläutert, welches ein grundlegender Algorithmus, der für diese Arbeit zugrunde liegender Implementierung ist.
Zusammenhängend mit Raytracing wird im Abschnitt vier, die Verwendung des Raytracers für das Volumenrendering erläutert.
Abschnitt fünf diskutiert die Auswahl des für diese Arbeit zugrunde liegenden Eyetrackers und dessen Verwendung für die Erfassung des fovealen und peripheren Bereichs.
Aus Performanzgründen kann es hilfreich sein für Berechnungen auf einer GPU, die Architektur der GPU zu betrachten und unter Umständen Algorithmen für eine bessere Effizienz anzupassen.
Diese Thematik wird in Abschnitt sechs behandelt.
\todo{Beschreibung der Grundlagen überarbeiten}

\section{Related Work}
Wahrnehmungsorientiertes Volumenrendering ist kein absolut neues Arbeitsgebiet und ist schon Teil einiger wissenschaftlicher Arbeiten gewesen \todo{Beispiele}.
Da diese Arbeit auf zwei trennbare Aspekte beruht, unterteile ich den Related Work Abschnitt in zwei Teile.
Der erste Abschnitt bezieht sich auf Arbeiten im Bereich des wahrnehmungsorientierten Renderings mit dem Ziel, die Performanz einer Anwendung zu steigern.
Die Ansätze hier beziehen sich meist darauf, dass die Qualität der Darstellung im peripheren Bereich der visuellen Wahrnehmung, gesenkt wird und so für die Berechnung eines Bildes weniger Rechenleistung aufgewendet werden muss.
\todo{Was genau ist mit Performanz gemeint?}
Der zweite Abschnitt bezieht sich auf Arbeiten zu wahrnehmungsorientiertem Volumenrendering, mit dem Ziel, die Qualität der Darstellung zu erhöhen.
Dabei werden vor allem Ansätze zur geschickten Anpassung von Parametern einer Transferfunktion vorgestellt, die dem Betrachter ein insgesamt besseres Verständnis der Volumendaten ermöglichen soll.
\subsection{Performanz- und Wahrnehmungsorientiertes Volumenrendering}
\todo{Related Work dafür}
\subsection{Wahrnehmungsorientiertes Volumenrendering zur Qualitätssteigerung}
\todo{Related Work dafür.}

\section{Sehapparat}
\todo{Grundlagen zum menschlichen Sehapparat hier.}

\section{Raytracing}
\todo{Grundlagen zu Raytracing hier.}

\section{Volumenrendering und Transferfunktion}
\todo{Grundlagen zum Volumenrendering und der zugehörigen Transferfunktion hier.}

\section{Eyetracking}
\todo{Grundlagen zum Eyetracking hier.}

\section{GPU Architektur}
\todo{Grundlagen zur GPU Architektur hier.}
