% Grundlagen für Arbeit (Related Work), Sehapparat, Raytracing, Volumenrendering, Eyetracking, GPU Architektur (Warps usw.) %

\chapter{Grundlagen}
\label{chap:k2}
Dieses Kapitel beschäftigt sich mit den Grundlagen für diese Arbeit.
Der erste Abschnitt des Kapitels beschäftigt sich mit verwandten Arbeiten zum Thema Wahrnehmungsorientiertes Volumen-Rendering.
Der zweite Abschnitt handelt über die Grundlagen des menschlichen Sehapparates.
Hier werden die Fähigkeiten und Limitierungen der visuellen Wahrnehmung des Menschen diskutiert.
In Abschnitt drei, wird die Funktionsweise von Raytracing erläutert, welches ein grundlegender Algorithmus, der für diese Arbeit zugrunde liegender Implementierung ist.
Zusammenhängend mit Raytracing wird im Abschnitt vier, die Verwendung des Raytracers für das Volumenrendering erläutert.
Abschnitt fünf diskutiert die Auswahl des für diese Arbeit zugrunde liegenden Eyetrackers und dessen Verwendung für die Erfassung des fovealen und peripheren Bereichs.
Aus Performanzgründen kann es hilfreich sein für Berechnungen auf einer GPU, die Architektur der GPU zu betrachten und unter Umständen Algorithmen für eine bessere Effizienz anzupassen.
Diese Thematik wird in Abschnitt sechs behandelt.


\section{Related Work}
Related Work hier.

\section{Sehapparat}
Grundlagen zum menschlichen Sehapparat hier.

\section{Raytracing}
Grundlagen zu Raytracing hier.

\section{Volumenrendering}
Grundlagen zum Volumenrendering hier.

\section{Eyetracking}
Grundlagen zum Eyetracking hier.

\section{GPU Architektur}
Grundlagen zur GPU Architektur hier.