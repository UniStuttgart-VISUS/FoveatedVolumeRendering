% Einleitung - Motivation - Problemstellung / Aufgabenstellung - Zeitplanung %
\chapter{Einleitung}\label{chap::intro}
Das visuelle Wahrnehmungssystem des Menschen hat gewisse Limitierungen, die im wahrnehmungsorientierten Rendering gezielt ausgenutzt werden können.
So kann die visuelle Wahrnehmung des Menschen grob in foveales und peripheres Sehen unterteilt werden.
Das foveale Sehen ist im Zentrum des Blickpunkts und umfasst nur einen sehr kleinen Bereich, in dem man scharf und detailliert sehen kann.
Das periphere Sehen hingegen ist unscharf und eher ungenau, wobei die Sehschärfe mit der Distanz zum Zentrum  des Blickpunkts immer weiter abnimmt.
Im wahrnehmungsorientierten Rendering wird sich unter Anderem diese Eigenschaft zu Nutze gemacht, um das Leistungsverhalten von bilderzeugenden Anwendungen zu verbessern.
Dafür wird die Bildqualität im peripheren Bereich gesenkt, wodurch die Berechnungsdauer eines Bildes reduziert wird.
Da die visuelle Wahrnehmungsmöglichkeiten im peripheren Bereich sowieso abnehmen, wird die Reduzierte Bildqualität im besten Fall von dem Nutzer nicht wahrgenommen.