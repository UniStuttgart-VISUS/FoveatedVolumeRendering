% Implementation: Umsetzung, Probleme, Lösungen
\chapter{Implementierung}\label{chap::impl}
Das Implementierungskapitel enthält eine genauere Beschreibung der Implementierung der Arbeitspakete aus Abschnitt \ref{sec::workpacks}.
Die Beschreibungen sollen einen Überblick geben, wie die Arbeitspakete umgesetzt wurden ohne zu stark ins Detail zu gehen.

\section{Strukturelle Modifikationen}\label{sec::sm}
Das Projekt wurde strukturell modifiziert und für die Integration der Arbeitspakete vorbereitet.
So wurde im Volumerenderwidget eine globale Variable erstellt, welche den Wert der aktuellen Raycast Methode beinhaltet.
In der paintGL() Methode wurde dann über diese Variable in einem switch-case die jeweilige Methode aufgerufen, die die Berechnung des aktuellen Raycasts durchführt.
Da die Raycasts durch einen OpenCL Kernel berechnet werden und nicht direkt gerendert werden, werden die Ergebnisse immer in einer Textur abgespeichert, welche durch OpenGL auf ein Fullscreen-Quad gezeichnet wird.
Da die MDC Methode zwei verschiedene Texturen verwendet, wurde dementsprechend der Fragment-Shader durch ein switch-case ergänzt, welches für jede Raycast Methode unterschiedliche Anweisungen ausführen kann.

Die Kernfunktion des Raycasts, die Strahlverfolgung, bleibt für alle Raycast Methoden die gleiche.
Daher wurde dieses Prinzip im OpenCL Kernel auch angewandt.
Ein switch-case, relativ am Anfang des Raycast Kernels, führt je nach aktueller Raycast Methode die entsprechenden Operationen aus.
Dadurch konnte für alle Raycast Methoden der gleiche Raycast Kernel verwendet werden, welcher lediglich mit unterschiedlichen Parametern ausgeführt wurde.

Die GUI des Projekts wurde mit einigen Funktionen ergänzt über die bestimmte Parameter verändert werden konnten.
Dies ermöglichte insbesondere die einfache Umstellung der Raycast Methoden zur Laufzeit des Projekts.

\section{Implementierung der Arbeitspakte}\label{sec::ida}
Im folgenden werden die genaueren Umsetzungen der Arbeitspakete vorgestellt, insbesondere die Implementierung des MDC und DDC Raycasts sowie die Anpassung der Strahlabtastrate.
Zusätzlich wird beschrieben, wie die Messungen im Projekt vorbereitet wurden, so dass Messwerte erstellt werden konnten.

\subsection{Simulieren der Blickposition}\label{sec::ida::sdb}
Da die Einbindung des Eyetrackers für das oberflächliche Testen der Raycast Methoden nicht notwendig ist, wurde die Blickposition vorerst mit der Mausposition simuliert.
Das Volumerenderwidget verfügt über eine Callback-Methode, die auf Mausbewegungen reagiert.
Diese Methoden wurde sich zu Nutze gemacht, so dass durch einen Aufruf dieser Methode, aufgrund einer Veränderung in der Mausposition, sofort die aktuelle Position des Mauszeigers bezüglich des Volumerenderwidgets in einer globalen Variable abgespeichert wurde.
Damit jede Raycast Methode diese auch zu Verfügung hat, wurde diese zu Beginn der paintGL() Methode dem OpenCL Raycast Kernel übergeben. (Abschnitt \ref{sec::workpacks::sdb})

\subsection{Reduzierung der Strahlabtastrate im fovealen Bereich}\label{sec::ida::rdsifb}
Die Reduzierung der Strahlabtastrate wurde im Raycast Kernel implementiert.
Nachdem alle Operationen der jeweiligen Raycastmodifikationen ausgeführt wurden, wurde für jedes Work-Item die Distanz der zugehörigen Bildkoordinate des entsprechenden Strahls zu der Mausposition berechnet.
Im MDC Raycast lag die Bildkoordinate nur normalisiert vor und wurde dementsprechend vor der Distanzberechnung umgerechnet.

Sei nun $\vec{r}$ ein 2D-Vektor, der die x- und y-Position der Bildkoordinate eines Strahls enthält und $\vec{m}$ ein 2D-Vektor, der die Bildkoordinaten der Mausposition enthält.
Die Distanz zwischen der Mausposition und einem Strahl wurde wie folgt berechnet: $d = \texttt{length(}\vec{r}-\vec{m}\texttt{)}$.
Die Funktion \texttt{lenght()} ist eine \emph{Built-In OpenCL Function} und berechnet die Länge eines Vektors.
Mit der Distanz $d$ zwischen einem Strahl und der Mausposition wurde ein Faktor $sf$ berechnet: $sf = 1,0 - d  / ib$.
Hierbei bezeichnet $ib$ die maximale Distanz zwischen zwei Bildkoordinaten, also die Länge der Diagonale des Bildes.
Ist $d < 10$ wird die Strahlabtastrate $sr$ nicht verändert.
Ansonsten berechnet sich die neue Strahlabtastrate mit $sr = \texttt{max(}sf * sr\,t\texttt{)}$. Mit $t = 0,25$ hat sich als guter Schwellwert erwiesen, der auch bei großen Distanzen zwischen einem Strahl und der Mausposition kaum Artefakte generiert.