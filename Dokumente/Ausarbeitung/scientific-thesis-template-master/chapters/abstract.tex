Diese Arbeit handelt um Wahrnehmungsorientiertes Volumen-Rendering. Das menschliche Auge ermöglicht dem Menschen seine visuelle Wahrnehmung, welche in foveales und peripheres Sehen unterteilt werden kann. Das foveale Sehen ist detailliert, scharf und farbig und befindet sich im Zentrum des visuellen Wahrnehumgsbereiches, während das periphere Sehen sich außerhalb des Zentrums befindet und unschärfer und weniger farbig ist. In dieser Arbeit wird speziell diese Eigenschaft des menschlichen Sehapparates im Zusammenhang mit Volumen-Rendering untersucht. Dafür werden unterschiedliche Ansätze betrachtet, die Bildqualität im peripheren Bereich zu senken, um die Performanz des Volumen-Renderings zu erhöhen und gleichzeitig die Qualität der Darstellung zu erhalten oder zu verbessern. Die Daten zur Ermittlung des fovealen beziehungsweise peripheren Bereichs werden mit einem Eye-Tracking Gerät gemessen und fließen direkt in die Darstellung ein. 