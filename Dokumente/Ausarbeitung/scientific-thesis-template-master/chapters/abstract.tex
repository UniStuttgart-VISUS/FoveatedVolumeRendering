Das menschliche Auge ermöglicht dem Menschen seine visuelle Wahrnehmung, welche in foveales und peripheres Sehen unterteilt werden kann.
Das foveale Sehen ist detailliert, scharf und farbig und befindet sich im Zentrum des visuellen Wahrnehumgsbereiches.
Im Gegensatz dazu ist das periphere Sehen der Bereich außerhalb des Zentrums, welcher unschärfer und weniger farbig wahrgenommen wird.
In dieser Arbeit wird speziell diese Eigenschaft des menschlichen Sehapparates im Zusammenhang mit Volumen-Rendering und Eyetracking untersucht.
Dies hat den Hintergrund, dass Volumen-Rendering an sich berechnungsintensiv ist und die Anforderungen an die Hardware durch immer höhere Auflösungen und geforderten Bildwiederholungsraten dauerhaft steigen.
Dafür werden zwei unterschiedliche Ansätze, die Reduzierung der Strahl- und die Reduzierung der Bildabtastrate, in eine bestehende Volumen-Rendering Anwendung implementiert und untersucht.
Damit wird beabsichtigt, die Bildqualität im peripheren Bereich zu senken, so dass die Performanz der Anwendung erhöht wird und gleichzeitig die wahrgenommene Qualität der Darstellung erhalten bleibt.
Die Implementierungen der unterschiedlichen Ansätze haben eine deutlich verbesserte Performanz bei einer kaum wahrnehmbaren Veränderung der Bildqualität im peripheren Bereich des Sehens ergeben.
Die Analyse der Untersuchungsergebnisse führten zu dem Schluss, dass sich wahrnehmungsorientierte Methoden für Volumen-Rendering besonders gut eignen.