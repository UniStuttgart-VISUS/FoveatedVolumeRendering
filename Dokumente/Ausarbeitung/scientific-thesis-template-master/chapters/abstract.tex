Das menschliche Auge ermöglicht dem Menschen seine visuelle Wahrnehmung, welche in foveales und peripheres Sehen unterteilt werden kann.
Das foveale Sehen ist detailliert, scharf und farbig und befindet sich im Zentrum des visuellen Wahrnehumgsbereiches, während das periphere Sehen sich außerhalb des Zentrums befindet und unschärfer und weniger farbig ist.
In dieser Arbeit wird speziell diese Eigenschaft des menschlichen Sehapparates im Zusammenhang mit Volumen-Rendering und Eyetracking untersucht.
Dies hat den Hintergrund, dass Volumen-Rendering an sich berechnungsintensiv ist, sowie die Anforderungen an die Hardware durch immer höhere Auflösungen und geforderten Bildwiederholungsraten dauerhaft steigen.
Dafür werden zwei unterschiedliche Ansätze, die Reduzierung der Strahl- und die Reduzierung der Bildabtastrate, in eine bestehende Volumen-Rendering Anwendung implementiert und untersucht, um die Bildqualität im peripheren Bereich mit dem Ziel zu senken, dass die Performanz des Volumen-Renderings erhöht wird und gleichzeitig die wahrgenommene Qualität der Darstellung erhalten bleibt.
Die Implementierungen der unterschiedlichen Ansätze haben eine deutlich verbesserte Performanz bei einer kaum wahrnehmbaren Veränderung der Bildqualität im peripheren Bereich des Sehens ergeben.
Daher konnte der Entschluss gezogen werden, dass wahrenehmungsorientierten Methoden sich für das Volumen-Rendering eignen.