% Entwurf: Projekt, Raytracer, (Auswahl des) Eyetracker, Vorüberlegungen zur Umsetzung / Integration, Grundidee zur Umsetzung
% Entwurfsziele an die Umsetzungs
\chapter{Entwurf}\label{chap::design}
Die Implementierung dieser Arbeit beruht auf einem Visual Studio Projekt zum Volumen Rendering von Valentin Bruder.
Der erste Abschnitt des Entwurfskapitels beinhaltet den Ausgangspunkt der Implementierung dieser Arbeit in Form einer Beschreibung des ursprünglichen Projekts.
Dies umfasst die allgemeine Architektur der Anwendung und die Umsetzung des Volumenrenderings durch eine Implementierung eines Raycasters als OpenCL Kernel.
Der zweite Abschnitt des Kapitels umfasst Überlegungen für die Erweiterung des Projektes, bezüglich des Ausgangspunktes wie er im ersten Abschnitt beschrieben wurde und dem Ziel dieser Arbeit, sowie die daraus entstandenen Arbeitspakete und die Ansätze für ihre Integration in das bestehende Projekt.

\section{Projekt}\label{sec::proj}
\todo{Allgemeine Visual Studio Projektbeschreibung. Architektur und Funktionsweise des Projekts.}

\subsection*{Raycaster}
\todo{Beschreibung der Funktionsweise des Raycasters für das Volumenrendering in dem Projekt.}

\section{Arbeitspakete und Integration}\label{sec::workpacks}
\todo{Vorüberlegungen zur Umsetzung und daraus entstandene Arbeitspakete beschreiben.}