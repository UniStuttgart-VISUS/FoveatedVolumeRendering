% Entwurf: Projekt, Raytracer, (Auswahl des) Eyetracker, Vorüberlegungen zur Umsetzung / Integration, Grundidee zur Umsetzung
% Entwurfsziele an die Umsetzungs
\chapter{Entwurf}\label{chap::design}
Die Implementierung dieser Arbeit beruht auf einem Visual Studio Projekt zum Volumen Rendering von Valentin Bruder.
Der erste Abschnitt des Entwurfskapitels beinhaltet den Ausgangspunkt der Implementierung dieser Arbeit in Form einer Beschreibung des ursprünglichen Projekts.
Dies umfasst die allgemeine Architektur der Anwendung und die Umsetzung des Volumenrenderings durch eine Implementierung eines Raycasters als OpenCL Kernel.
Der zweite Abschnitt des Kapitels umfasst Überlegungen für die Erweiterung des Projektes, bezüglich des Ausgangspunktes wie er im ersten Abschnitt beschrieben wurde und dem Ziel dieser Arbeit, sowie die daraus entstandenen Arbeitspakete und die Ansätze für ihre Integration in das bestehende Projekt.

\section{Projekt}\label{sec::proj}
% QT
Das Visual Studio Projekt, welches als Ausgangspunkt dient, ist eine auf QT basierende Anwendung.
QT ist ein vollständiges Cross-Platform Softaware Development Framework, welches in c++ entwickelt wurde, und ermöglicht die einfache Erstellung von Anwendungen mit Benutzeroberflächen und bietet außerdem eine Vielzahl von Bibliotheken, die für eine leichtere und schnellere Entwicklung von Programmen genutzt werden können.

% Struktur von QT
Die Hauptelemente für die QT Benutzeroberfläche sind QT Widgets.
Widgets können Daten darstellen und Nutzereingaben erkennen.
Außerdem stellt ein Widget selbst ein Container für weitere Widgets dar, welche in diesem gruppiert werden.
Innerhalb eines Widgets können Elemente platziert werden, welche Informationen, mögliche Operationen oder Nutzereingaben repräsentieren.
Für die bequeme Erstellung der grafischen Oberfläche bietet QT die Anwendung QT Designer.
Der QT Designer ermöglicht es, Widgets und andere Bausteine der grafischen Oberfläche per Drag und Drop anzuordnen.
Die durch den QT Designer definierte Oberfläche kann abgespeichert werden und wird von QT verwendet, um eine c++ Header File zu erstellen.
Dies ermöglicht es die verschiedenen Elemente der grafischen Oberfläche an die Logik der Anwendung zu binden.
\todo{Quelle zu QT einfügen.}

% Struktur des Projekts (verschiedene Widgets)
Das Projekt ist dementsprechend in c++ geschrieben und die grafische Oberfläche wurde mit Hilfe des QT Designers erstellt.
Die Oberfläche selbst hat eine Menüleiste, die es unter Anderem ermöglicht, Volumendaten oder Transferfunktionen zu laden oder auch aktive Transferfunktionen zu speichern sowie das Erstellen eines Screenshots des zuletzt berechneten Bildes.
Den Großteil der grafischen Oberfläche wird durch das Volumenrenderwidget ausgefüllt.
Das Volumenrenderwidget ist ein QT OpenGL Widget und kann für das Darstellen von OpenGL Grafiken verwendet werden.
Die berechneten Grafiken werden mit Hilfe des Volumenrenderwidgets dargestellt.
Neben dem Volumenrenderwidget gibt es noch ein Widget, welches in drei weitere Widgets unterteilt ist, mit denen Parameter für das Volumenrendering gesetzt werden können.
Das erste von ihnen ermöglicht das variieren der Abtastrate im Bildraum, also die Anzahl der Strahlen, die ausgesendet werden, sowie das Setzen der allgemeinen Abtastrate der ausgesendeten Strahlen.
Außerdem können hier weitere Rendering Parameter festgelegt werden, wie die Hintergrundfarbe oder ob Voxel beim Abtasten interpoliert werden.
Das zweite Widget ist ein Farbenrad, welches für die einfache Auswahl der Farben einzelner Kontrollpunkte der Transferfunktion verwendet werden kann.
Das dritte Widget ermöglicht schließlich das Setzen von Kontrollpunkten der Transferfunktion innerhalb eines Diagramms.
Die x-Richtung gibt die Dichte, auf die sich ein Kontrollpunkt bezieht an.
Die y-Richtung gibt seinen Opazitätswert an.
Die Werte zwischen zwei Kontrollpunkten werden entweder linear oder quadratisch interpoliert.
Daher gibt es immer mindestens einen Kontrollpunkt für den Dichtewert null und einen Kontrollpunkt für den Dichtewert eins.
\todo{Bitte auf Richtigkeit von diesem Teil prüfen, wie: Interpolation von Voxel und Einstellen der Transferfunkton.}

% Speziell VolumeRenderWidget: OpenCL, OpenGL Host Code
Das Volumenrenderwidget ist ein OpenGL Widget und für die Darstellung der berechneten Bilder zuständig.
Das QT Framework erlaubt es, das Widget in c++ Code mit Logik zu verknüpfen.
Dafür existiert in dem Projekt eine Klasse VolumeRenderWidget, welche von QOpenGLWidget erbt.
Die grundsätzliche Funktionalität zum Rendern in diesem Widget wird durch die Methode \texttt{paintGL()} ausgeführt.
Innerhalb dieser Methode wird der Code geschrieben, der für die Darstellung des Bildes nötig ist.
Das Darstellen auf dem Bildschirm beziehungsweise in dem Widget wird durch OpenGL realisiert.
OpenGL zeichnet dabei aber lediglich eine durch einen OpenCL Kernel zuvor generierte Textur auf ein Fullscreen Quad.
Das Management von OpenCL wird hier durch ein Objekt der Klasse \texttt{volumerendercl} geregelt.
Innerhalb der \texttt{paintGL()} Methode wird eine Methode dieses Objekts zum Starten des OpenCL Kernels für den Raycast des Volumenrenderings aufgerufen.
Das \texttt{volumerendercl} Objekt regelt die Handhabung der verschiedenen Parameter für den Kernel und die Unterteilung der übergebenen Anzahl an Strahlen in x- und y-Richtung, welche der Anzahl zu startenden Work-Items entspricht, in Work-Groups.
Außerdem startet es den Kernel, synchronisiert einen gemeinsamen Stopp und speichert die benötigte Zeit der letzten Ausführung des Kernels.
Die berechneten Werte der einzelnen Work-Items können bei der Ausführung des Kernels direkt in die OpenGL Textur geschrieben werden.
Daher kann nach der Ausführung der aufgerufenen Methode des \texttt{volumerendercl} Objekts die Textur direkt gezeichnet werden.
Mit Hilfe von QT Funktionen und der Information über die Ausführungszeit des Kernels werden anschließend noch ein paar Overlays gezeichnet.
Unter Anderem eine Anzeige der ungefähren möglichen Anzahl an Bildern pro Sekunde der letzten Ausführungen, um die Ausführungsdauer des Kernels abschätzen zu können.

% Raycaster: Raycastkernel
\subsection*{Raycaster}
% Raycaster: wichtige Parameter: in, out; Ungefährerer Aufbau: Sampling Loop durch das Volumen;
Der eigentliche Raycast passiert in einem OpenCL Kernel.
Die OpenCL Objekte werden von dem \texttt{volumerendercl} Objekt gehandhabt, dessen Methoden innerhalb der \texttt{paintGL()} Methode aufgerufen werden.
Das \texttt{volumerendercl} Objekt regelt auch die Übergabe der Parameter an den Raycast Kernel.
Dies sind Parameter wie Volumendaten, Transferfunktionswerte und Strahlabtastrate zum lesen, sowie eine 2D-Textur zum schreiben für die berechneten Farbwerte der einzelnen Work-Items.
Jedes Work-Item besitzt eine 2D-ID, die einer Position in der Ausgabetextur zugewiesen bekommt.
Ein Work-Item ist für das Abtasten eines Strahls verantwortlich.
Ein Strahl hat als Ursprung die Position der Kamera und entsprechend seiner ID, beziehungsweise Texturkoordinaten, wird seine Richtung bestimmt.
Abhängig von der Abtastrate des Strahls, berechnet sich die Schrittgröße für das Abtasten. 
Ausgehend von dem Schnittpunkt des Strahls mit der Bildebene wird nun in einer Schleife der Strahl schrittweise abgetastet.
Dabei wird für jeden Schritt die aktuelle Position bestimmt, welche normiert und dann dafür genutzt wird, um in dem 3D-Volumen Objekt den Dichtewert für diese Position zu bestimmen.
Mit dem Dichtewert und den Daten der Transferfunktion wird anschließend ein Farbwert berechnet.
Dieser Farbwert wird mit den bisherigen gesammelten Farbwerten des Strahls verrechnet, so dass am Ende der Abtastschleife ein einziger Farbwert für den Strahl existiert.
Der Farbwert wird zum Schluss an die entsprechende Texturkoordinate in der Ausgabetextur gespeichert.

%  Raycaster: Spezielle Eigenschaften, die aktiviert und deaktiviert werden können: ESS, Interpolation, AO


\todo{Beschreibung der Funktionsweise des Raycasters für das Volumenrendering in dem Projekt.}

\section{Arbeitspakete und Integration}\label{sec::workpacks}
\todo{Vorüberlegungen zur Umsetzung und daraus entstandene Arbeitspakete beschreiben.}