% % Fazit der Arbeit
\chapter{Fazit}\label{chap:zusfas}
% \todo{Fazit wertend (was hat gut und schlecht funktioniert aber auf den gesamten Zeitverlauf der Arbeit bezogen).}
Die Einarbeitung in das Projekt war relativ aufwendig und es hat daher eine gewisse Zeit gebraucht, bis sich ein Überblick über die Struktur des Projektes gebildet werden konnte.
Dies ist im Hinblick auf ein umfangreiches Projekt nicht verwunderlich, wurde aber auch dadurch erschwert, dass unter Anderem mit verschiedenen ungewohnten Programmiersprachen, Programmierumgebungen und Programmierbibliotheken sich auseinander gesetzt werden musste.
Die Auseinandersetzung mit diesen hat aber viele neue Einblicke in unterschiedliche Bereiche ermöglicht, die sich im Laufe dieser Arbeit sich als sehr hilfreich repräsentiert haben und als Grundlage für neue Arbeiten dienen werden.
So haben sich vor allem Einblicke in die GPU Architektur als sehr interessant und hilfreich bei der Implementierung der Implementierung der Raycasts, besonders des DDC Raycasts erwiesen.
Zum Beispiel wurde aufgrund der Einblicke in die Grundlagen der GPU Architektur im Laufe dieser Arbeit die Implementierung des DDC Raycasts strukturell verändert, wodurch sich die Performanz weiter verbesserte.
Trotzdem, wie in der Diskussion (Abschnitt \ref{sec::disc}) sich schon herausgestellt hat, muss man bei der Programmierung für die GPU auf bestimmte Aspekte achten und meistens ist eine Optimierung immer noch möglich.
Die Auseinandersetzung mit den Grundlagen und Limitierungen des Sehapparates (Abschnitt \ref{sec::eye}) war für das Verständnis des Hauptziels dieser Arbeit und für die Umsetzung der Implementierungen wichtig und hat die Art und Weise, wie die Implementierungen entworfen wurden, beeinflusst.
Allerdings wurden die Parameter, wie die unterschiedlichen Auflösungen und die unterschiedlichen Ausmaße der Bereiche in den verschiedenen Raycast Methoden, nach der eigenen Wahrnehmung eingestellt.
Diese Parameter könnten für weitere Arbeiten genauer untersucht werden, um die optimalen Werte zu finden.
Zum Beispiel die könnte untersucht werden, wie es sich auswirkt, wenn die Größe der inneren Ellipse des DDC Raycasts auf die genaue Größe der Fovea auf dem Bildschirm angepasst wird und wie es wahrgenommen wird, wenn die innere Ellipse stattdessen kleiner oder größer dargestellt wird.

Im Verlauf der Arbeit wurde klar, dass die Umsetzung des Volumenrenderings durch einen Raycast sehr gut möglich ist und auch viele Erweiterungen und Modifikationen des Volumenrenderings, wie wahrnehmungsorientierte Methoden, zum Beispiel das Reduzierung der Strahlabtastrate im peripheren Bereich, oder das Anwenden einer Tranferfunktion, dadurch einfach umgesetzt werden kann.
Wahrnehmungsorientiertes Rendering ist aber nicht auf das Volumenrendering und Raycasts beschränkt.
Die Referenzliteratur in Abschnitt \ref{ss::pfwov} zeigte diesbezüglich Arbeiten, die wahrnehmungsorientierte Methoden in Anwendungen anwenden, die eine Grafikpipeline zur Berechnung verwenden.
% Einen Entwurf anzufertigen mit Arbeitspaketen, die durchgesetzt werden sollen, war hilfreich da man sich daran halten konnte und ermöglichte eine gewisse einteilung der arbeit
% Bezogen auf die Ergebnisse hat es gezeigt, dass wahrnehmungsorientierte methoden durchaus einen performanzgewinn bringen können ohne die bildqualität zu beeinträchtigen.
% Alleine die Anpassung der Strahlabtastrate hat kaum wahrnehmbare Bildqualitative Veränderungen nach sich gezogen und trotzdem die Ausführungszeit spürbar verbessert.
% Die anpassungen der Bildabtastrate gab es zwei Methoden, mdc ddc
% mdc der einfachere umgesetzte ansatz und ging von der implementierung her auch deutlich schneller
% trotzdem mdc it sar insgesamt am besten abgeschnitten, sowohl performanz als auch deutlich besser in der bildqualität als ddc
% den ansatz, der bei mdc gegangen wurde könnte man daher weiter ausbauen, z.B. bildabtastrate im äußeren bereich weiter verringern und u.U. einen dritten bereich einfügen, solange der Overhead dadurch nicht zu groß wird, da bisher im vergleich mdc hauptsächlich wegem dem overhead bessere werte gezeigt hat
% ddc war deutlich zeitaufwendiger zu implementieren, da index hinund her gemappt wurden und anschließend noch eine interpolation benötigt wurde
% ddc hat bessere spitzenwerte aber ist von der Performanz her aufgrund des overheads schlechter als mdc
% auch die Bildqualität leidet unter der niedrigen Bildabtastrate, wird aber aufgrund der limitationen des visuellen wahrnehmungssystems bei der verwendung eines eyetrackers lange nicht so wahrgenommen, wie wenn man nur das statische Bild betrachtet, ohne einen Eyetracker
% ddc ist wie in diskussion schon erwähnt ausbaufähig und kann vermutlich bei einer besseren umsetzung deutlich effizienter werden und aufgrund des einzelnen raycasts auch vom overhead schneller werden.
% Im Vergleich zur Referenzliteratur, in welcher deutlich höhere Faktoren in der Performanzgewinnung erfahren wurden, liegt dies unter anderem auch daran, dass die raycast methoden vor allem die ddc methode in der implementierung möglichst variabel gehalten wurde und auch daran, dass wie schon erwähnt in den implementierungen hier noch genügend verbesserungsmöglichkeiten existieren.
% Ein interessanter Ansatz für die Zukunft, wäre eine implementierung die wie eine Art Linse (verweis überlegung im arbeitspaket), die die Strahldichte gleichmäßig reduzieren kann
% Bezogen Auf die Arbeit mit dem Eyetracker war es sehr interessant damit zu arbeiten.
% Dass die api eine validierung der augendaten zur verfügung stellt hat sehr geholfen. Insgesamt ist aber bei der Arbeit mit dem Eyetracker noch aufgefallen, dass es entweder durch die Tremor-Bewegungen des Auges oder durch Messungenauigkeiten des Eyetrackers zu einem Flackern kommt. Da das Bild auf das Flackern reagiert ziehen sich veränderte Eigenschaften die Aufmerksamkeit auf sich und dadurch wirkt das Bild ein wenig störend. Für eine weitere Arbeit mit Eyetracking Daten sollten diese gefiltert oder geglättet werden.
% Bezogen auf die Erstellung und Darstellung der Messwerte kann man sagen, dass dies doch noch ein erheblicher Aufwand war, die Messungen in dem Projekt umzusetzen und aus den Messdaten grafische repräsentationen zu bilden, die verständlich sind.
% Schlussendlich konnten die Messmethoden so implementiert werden, dass diese reproduzierbar sind.
% Für die Messung war es auch sinnvoll, verschiedene Mauspositionen bei der Darstellung eines Volumens zu betrachten.
% Die Darstellung durch Heatmaps veranschaulicht die vor- und nachteile der methoden ddc und mdc bei der betrachtung eines volumens und die darstellung der boxplots ermöglichte einen noch genaueren und verständlicheren vergleich.
% Abschließend ist zu sagen, dass sowohl die vorgestellten Methoden als auch wahrnehmungs-orientiertes Rendering hohes Potential hat und aufgrund der größeren Bildschirme und Pixeldichten vor allem auch in VR-Anwendungen, die nicht immer mit high-end hardware ausgestattet sind, dies auch in zukunft einen wichtigen aspekt haben wird.
% Ausserdem ist die Performanz nicht der einzige Grund für wahrnehmungsorientiertes Rendering. Wie auch in der Referenzliteratur beschrieben, kann dieses auch für Informativere Darstellungen sorgen. Oder auch die Augenbewegungen für die Steuerung von Anwendungen verwenden.